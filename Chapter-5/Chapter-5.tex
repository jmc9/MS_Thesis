\chapter{Discussion}
\label{chap-five}

In this chapter the work presented in this paper is summarized and final conclusions are drawn. There is a discussion on what work still needs to be done on the ROMs presented, and the research that is planned for the future.

\ind In this study two new reduced-order models were presented for solving 1D TRT problems. Both are formed on the bases of the multilevel nonlinear iterative-projective method known as the multilevel quasidiffusion method (Sec. \ref{sec:qd_roms}) and the data-driven methodology known as the proper orthogonal decomposition (Sec. \ref{sec:pod}). These ROMs avoid use of the high-order radiative-transfer equation \eqref{General_RTg_eqn} by making approximations for the QD factors with the POD, and the second grey ROM further avoids use of the multigroup low-order QD equations \eqref{mqd_sys} by approximating its solution with the POD. 

\ind The first ROM, presented in Chapter \ref{chap-three} and referred to as the MLOQD-POD ROM, was able to successfully recreate the reference MLQD solution when using a full-rank POD representation of the reference QD factors. Compared to other well-known (classical) ROMs (Sec. \ref{sec:classical_roms}) the MLOQD-POD ROM gave significant increase in accuracy while using crude low-rank POD representations of the QD factors. The ROM was extended to solve problems with a different time step relative to that used to calculate the reference database of QD factors, which increased the errors found noticeably during the very early times of the problem. This was attributed to the rapid rate of change of the solution during this period of the problem dynamics, which are difficult to reproduce with linear interpolation. Near the equilibrium solution however, the accuracy of the ROM increased by several orders of magnitude. The error level was also found to saturate while using a considerably low-rank POD representation of the QD factors. The MLOQD-POD ROM was additionally parameterized with respect to the temperature of radiation moving into the problem domain. The results for this parameterization demonstrated performance that remained significantly ahead of those other classical ROMs, but whose error levels reached a saturation point at a rather low-rank POD representation of the QD factors. The accuracy of the ROM deteriorated as the solved problem's incoming radiation temperature diverged from the incoming radiation temperatures used to calculate the database.

\ind The results shown for this first ROM are promising and demonstrate that it is able to retain accuracy of its solution without solving the RT equation, and that it significantly outperforms other ROMs such as $P_1$ and diffusion. One of the main obstacles found while developing the ROM was obtaining a sufficiently low-rank approximation of all group QD factors. Some optically thick groups naturally demonstrated wave-like behavior at the early times of the problem which the POD struggled to recreate with very low-rank. There exist other methods of database decomposition that may alleviate this problem, such as the dynamic mode decomposition (DMD) \cite{schmid-2010,rgm-tsh-m&c2019} or shifted-POD \cite{reiss2-2018,reiss-2018} The POD was performed separately for each group QD factor database to enhance the accuracy of the ROM but there remains other avenues of applying the POD such as forming one database that includes all group QD factors and only performing the POD once. The accuracy seen for the parameterized ROMs and those solving a problem with a reduced time step length compared to what was used to calculate the database could have been limited by the linear interpolation scheme used to find unknown QD factors from the given databases. Higher order interpolation schemes remain to be investigated. Beyond this, the MLOQD-POD ROM is planned to be extended into 2D in space and for more general radiative-hydrodynamics problems that have dependence on material density and position. In 2D transport effects become more prolific, and the QD factors develops increased complexity. This will act as the next step to determine how well this ROM is able to handle more complicated transport and spatial effects. Radiative-hydrodynamic problems add extra equations to the system which must be coupled to the LOQD system and this will aid in demonstrating the advantage given by this ROM in multiphysical problems with more coupling effects than the TRT problem.

\ind The second ROM, presented in chapter \ref{chap-four} and referred to as the GLOQD-POD ROM, was not able to recreate the reference MLQD solution when using a full-rank POD representation of the reference QD factors and group energy densities. Using reference values for the group energy densities was the only way to recreate the reference solution. The POD was demonstrated to be unable to recreate the reference group energy densities due to numeric precision limitations, and this affected the accuracy of the ROM. An extra correction was added to the POD (Sec. \ref{bb_cor}) to mitigate this problem but was unable to fully fix it. The conclusion was drawn that the accuracy of this ROM should first be improved before extending it to parameterization as done for the first ROM, although results for these extensions were shown to demonstrate their behavior. Further analysis must be done to determine if the ROM is able to recreate the reference solution with a POD representation of the group energy densities, and if other methods of approximation are more suitable as in the DMD. If further development of the ROM proves it to be more accurate then further extensions are planned similar to those for the first ROM, such as extending into 2D in space and to general radiative-hydrodynamics problems.